\documentclass{article}
\usepackage[utf8]{inputenc}

\usepackage{amsmath}
\usepackage{amsfonts}
\title{MA581 Problem Set 4}
\author{Kyle Mickelson A4}

\date{March 4, 2022}
\begin{document}

\maketitle

\noindent\textbf{Problem 4.50}\\
a.) Let A denote the event that a 16-24 year old is involved in crash, B denote the event that a 25-34 year old is involved in crash, C denote event 35-64 year old involved in crash, and D denote the event that 65+ year old involved in crash. Assuming that these events are independent by context, we can see that:
    $$ P(A\cap\!B\cap\!C) = P(A)P(B)P(C) = (0.255)(0.238)(0.393) \approx 0.02385$$
b.) Of 3 crashes, 2 of drivers are 16-24 year old's and 1 is a 65+ years old.
Then, choose 2 of the 3 crashes to be 16-24 year olds and 1 to be a 65+ year old.

$$ \binom{3}{1}(0.255)^{2}(0.144) \approx 0.0281  $$
\noindent\textbf{Problem 4.55}\\
a.) By definition of mutually exclusive, we know that $A\cap B = \emptyset$. We also know that the joint probability of two events A and B is represented by $P(A\cap\!B)$. This then implies that $P(A\cap B) = P(\emptyset) = 0$.
\\\\
b.) From a.), we know that in order for two events to be mutually exclusive, their intersection must be 0, which means their joint probability must also be 0. From the definition of independent events, we know that $P(A\cap B) = P(A)P(B)$. Then, since we are given that $P(A),P(B)>0$, we know $P(A\cap\!B)>0$. Therefore the joint probability of A and B cannot be 0. 
\\\\
c.) A = \{ I am happy \} and B = \{ I am eating \}. These are not mutually exclusive, because A can definitely occur when B occurs. It is also not independent, because the event B increases the likelihood of event A occurring. 
\\\\
\noindent\textbf{Problem 4.56}\\
a.) The probability of getting a head on the first toss is obtained by adding the probabilities of getting a head on the first toss by the fair coin and by the unfair coin. Let $C_{1}$ denote the event that the fair coin is chosen and $C_{2}$ denote the event that the unfair coin is chosen.


$$ P(H) = P(C_{1})P(H\mid\!C_{1}) + P(C_{2})P(H\mid\!C_{2}) $$
$$ P(H) = (1/2)(1/2) + (1/2)(p) = 1/4 + p/2$$
b.) The probability of getting a head on the second toss is obtained by adding the probabilities of getting a head on the second toss by the fair coin and by the unfair coin. Similarly to part a.), we get $P(H)=1/4 + p/2$.
\\\\
c.) The probability of getting two heads is given by the following:
$$P(HH)=P(H\mid\!C_{1})P(HH\mid\!(H\mid\!C_{1}))+P(H\mid\!C_{2})P(HH\mid\!(H\mid\!C_{2}))$$
Then after pluggin in the values from a.) and b.), we find that $P(HH)=1/8 + p^2/2$.
\\\\
d.) The events 'first toss is a head' and 'second toss is a head' are independent if and only if $p = 1/2$.
Proof: Suppose the events 'first toss is a head' from a.) and the event 'second toss is head' from b.) are independent. Then, we know that $P(H_{1}\cap H_{2}) = P(HH) = P(H_{1})P(H_{2}) = (1/4 + p/2)^2 = 1/8 + p^2/2$. Then we can see that $p^2-p+1/4 = 0$, which factors into $(p+1/2)(p-1/2)=0$. Then since probabilities are greater than or equal to 0, we know that the only solution is that $p=1/2$. Thus the unfair coin must be a fair coin if independent events. 
Conversely, suppose that $p=1/2$. Then we can see that the event from a.), the event of 'first toss is head' has probability of 1/2. Similarly for b.), the event that the 'second toss is head' has a probability 1/2. Then, since the intersection of the events 'first toss is head' and 'second toss is head' is equal to the event in c.), that both are heads, we see that this probability is 1/4. Then, $P(H_{1} \cap H_{2}) = P(H_{1})P(H_{2}) = 1/4$. Therefore the events are independent.
\\\\
\noindent\textbf{Problem 4.69}\\
Let A denote the event that the person selected is accident prone, let B denote the event that the person selected is normal, and let C denote the event that a person had no accidents for a year. Then, since A and B form a partition on the sample space, we know from Bayes' rule that:

$$P(A\mid\!C) = \frac{P(A\cap C)}{P(C)} = \frac{P(A)P(C\!\mid\!A)}{P(A)P(C\!\mid\!A)+P(B)P(C\!\mid\!B)}$$
We are given that $P(C^{C}\!\mid\!A)=0.6 $,$P(C^{C}\!\mid\!B)=0.2 $, and $P(A) = 0.18$. Therefore, by the complementation rule, we know that $P(C\!\mid\!A)=0.4 $ and $P(C\!\mid\!B)=0.8 $. Substituting into the equation above we obtain $P(A\!\mid\!C)\approx 0.0989$.
\\\\
\noindent\textbf{Problem 4.74}\\
Let A denote the event that the fair coin is chosen and let B denote the event that the unfair coin is chosen. Then, since randomly choosing from the two coins, P(A) = P(B) = $1/2$.
\\\\
a.) The probability that fair coin is chosen give first toss is head and second toss is tail is given by:
$$ P(A\!\mid\!HT) =  \frac{P(A\cap HT)}{P(HT)}=\frac{1/8}{17/72} = 9/17$$
b.) The probability that fair coin is chosen give first toss is tail and second toss is head is given by:
$$ P(A\!\mid\!TH) = \frac{P(A\cap TH)}{P(TH)}=\frac{1/8}{17/72} = 9/17$$
c.) The probability that fair coin is chosen given that exactly one head is tossed in the two tosses is given by:
$$ P(A\!\mid\!\{exactly \;1\; H\}) = \frac{P(A\cap \{exactly \;1\; H\})}{P(\{exactly \;1\; H\})}=\frac{1/4}{17/36} = 9/17 $$









\noindent\textbf{Problem 5.8}\\
We are given that six men and five women apply and only three are accepted. Let $X$ denote the number of women accepted.
\\\\
a.) Since three applicants are accepted, the possible values of $X$ are no women, one woman, two women, and three women selected. $Range(X) = \{0,1,2,3\}$.
\\\\
b.) For the number of possible outcomes corresponding to each value of $X$, we have: 
$
\\
    \{X=0\} = \binom{6}{3} = 20
\\
    \{X=1\} = \binom{6}{2}\binom{5}{1} = 75
\\
    \{X=2\} = \binom{6}{1}\binom{5}{2} = 60
\\
    \{X=3\} = \binom{5}{3} = 10$
\\
\\
c.) The set $\{X\leq 1\}$ denotes the event that at most one woman is selected in the experiment. It is equal to the set $\{X=0\}\cup \{X=1\}$. So 20 + 75 = 95 different outcomes. 
\\\\
\noindent\textbf{Problem 5.14}\\
Proof: Assume by way of contradiction that there are more than $n-1$ values of $x \in \mathbb{R}$ that satisfy the inequality $P(X=x)> \frac{1}{n}$, where $n\in \mathbb{N}$. Then we can see that there would be $P(X=x)>1$, which we know to be impossible. This is the contradiction as required. Now, we can see that since there are at most $n-1$ values of $x\in\mathbb{R}$ that satisfy this inequality.

Then, since there can be at most $n-1$ values and $n\in \mathbb{N}$, we can say that all of the individual $x$ are finite and their union thus countable, represented by $\bigcup_{i=1}^{n-1}\{x_{i}\} = \{x\in \mathbb{R}: P(X=x)>0\}$. Therefore the set $\{x\in \mathbb{R}: P(X=x)\neq0\}$ is countable as desired. 
\\\\
\noindent\textbf{Problem 5.28}\\
Letting $X$ be the number of red balls drawn from the experiment, we can see that the range of $X$ is the set $Range(X) = \{0,1,2\}$, because we can either pick two red balls, one red ball, or no red balls.
\\\\
a.) For $p=0.5$, we obtain the probability mass function $P_{X}(x)=P(X=x)=$
\[   \left\{
\begin{array}{ll}
      \frac{3}{10} \;if\; x = 0 \\
      \\
      \frac{4}{10} \;if\; x = 1 \\
      \\
      \frac{3}{10} \;if\; x = 2 \\
\end{array} 
\right. \]
\\\\
b.) For $p=0.6$, we obtain the probability mass function $P_{X}(x)=P(X=x)=$
\[   \left\{
\begin{array}{ll}
      \frac{2}{25} \;if\; x = 0 \\
      \\
      \frac{10}{25} \;if\; x = 1 \\
      \\
      \frac{13}{25} \;if\; x = 2 \\
\end{array} 
\right. \]
\\



\noindent\textbf{Problem 5.46}\\
We are given that the sample gathered is of size $n=20$, with the probability that a voter votes "yes" given as 0.6, also that it is taken with replacement. Therefore we can approximate this experiment with a discrete random variable $X$ that denotes the number of voters saying "yes", where $X\sim B(n=20,p=0.6)$. To find the probability that there are more voters who intend to vote "yes", we can model the distribution as:
$$ P(X\geq11) = \sum_{x=11}^{20} \binom{20}{x}(0.6)^{x}(0.4)^{20-x} \approx 0.75534$$
\noindent\textbf{Problem 5.50}\\
Let $X$ be a discrete random variable denoting the number of 6's or 7's in the random decimal digits chosen. The success probability $p$ will be equal to 0.2 because for each decimal digit there are 10 available options, 0-9. We want the probability that there is at least one 6 or 7 in the decimal digit sequence to exceed 0.95. Therefore we have:
$$P(X\geq 1) = 1-P(X<1) = 0.95 $$
Since there are a finite number of trials each with equal probability of success, and each independent, $X\sim Bin(n,p)$. Then we have:
$$ P(X<1) = P(X=0) = \binom{n}{0}(0.2)^0(0.8)^n = (0.8)^n$$
Then if we plug this back into the previous equation, we find that:
$ 1-(0.8)^n = 0.95$ which becomes $(0.8)^n = 0.05$. Taking the natural logarithm of both sides shows us that $n=13.425$. Therefore the sequence must be at least 14 digits long for the probability of getting a 6 or 7 to exceed 0.95. 

\end{document}
